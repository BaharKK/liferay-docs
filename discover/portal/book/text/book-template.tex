
\documentclass[9pt,openright,twoside]{memoir}

\usepackage{ucs}
\usepackage[english]{babel}
\usepackage{fontspec}
\usepackage{graphicx}
\usepackage{calc}
\usepackage[xetex]{hyperref}
\usepackage{enumerate}
\usepackage{ctable}
%\usepackage[labelformat=empty,font=small]{caption}
\usepackage{titlesec}
\usepackage{float}
\usepackage{morefloats}
\usepackage{wrapfig}
\usepackage{longtable}
\usepackage{geometry}
\defaultfontfeatures{Ligatures=TeX}

\geometry{paperwidth=6in,paperheight=9in,
    hmargin={.75in,.75in},vmargin={.75in,.75in}}

\setmainfont{Alegreya}
\setsansfont{Cabin}
\newfontfamily\sectionfont{Cabin}
\newfontfamily\subsectionfont{Cabin}
\newfontfamily\subsubsectionfont{Cabin}
\newfontfamily\caption{Cabin}
\titleformat*{\section}{\Large\bfseries\sffamily\sectionfont}
\titleformat*{\subsection}{\bfseries\sffamily\subsectionfont}
\titleformat*{\subsubsection}{\itshape\subsubsectionfont}
\captionnamefont{\tiny\sffamily}
\captiontitlefont{\tiny\sffamily}

\makeatletter
\g@addto@macro\@verbatim\scriptsize
\makeatother

\newlength{\imgwidth}
\newlength{\drop}% for my convenience

\newcommand*{\titleGM}{\begingroup% Gentle Madness
\drop = 0.1\textheight
%\vspace*{\baselineskip}
\vfill
\hbox{%
\hspace*{0.2\textwidth}%
\rule{1pt}{\textheight}
\hspace*{0.05\textwidth}%
\parbox[b]{0.75\textwidth}{
\vbox{%
\vspace{\drop}
{\noindent\HUGE\bfseries Using\\[0.5\baselineskip]
Liferay Portal}\\[2\baselineskip]
{\Large\itshape A Complete Guide}\\[4\baselineskip]
{\Large THE LIFERAY DOCUMENTATION TEAM}\par
{\small Richard Sezov, Jr.}\par
{\small Jim Hinkey}\par
{\small Stephen Kostas}\par
{\small Jesse Rao}\par
{\small Cody Hoag}\par
{\small Nicholas Gaskill}\par
{\small Michael Williams}\par
\vspace{0.25\textheight}
{\noindent Liferay Press}\\[\baselineskip]
}% end of vbox
}% end of parbox
}% end of hbox
\vfill
\null
\endgroup}

\makeatletter
\newcommand\thickhrulefill{\leavevmode \leaders \hrule height 1ex \hfill \kern \z@}
\setlength\midchapskip{10pt}
\makechapterstyle{VZ14}{
  \renewcommand\chapternamenum{}
  \renewcommand\printchaptername{}
  \renewcommand\chapnamefont{\sffamily\Large\scshape}
  \renewcommand\printchapternum{%
    \chapnamefont\null\thickhrulefill\quad
    \@chapapp\space\thechapter\quad\thickhrulefill}
  \renewcommand\printchapternonum{%
    \par\thickhrulefill\par\vskip\midchapskip
    \hrule\vskip\midchapskip
  }
  \renewcommand\chaptitlefont{\sffamily\Huge\scshape\centering}
  \renewcommand\afterchapternum{%
    \par\nobreak\vskip\midchapskip\hrule\vskip\midchapskip}
  \renewcommand\afterchaptertitle{%
    \par\vskip\midchapskip\hrule\nobreak\vskip\afterchapskip}
}
\makeatother

\newcommand\scalegraphics[1]{%   
    \settowidth{\imgwidth}{\includegraphics{#1}}%
    \setlength{\imgwidth}{\minof{\imgwidth}{\textwidth}}%
    \includegraphics[width=\imgwidth]{#1}%
}


\usepackage{fancybox}
\newenvironment{roundedframe}{%
\def\FrameCommand{%
\cornersize*{20pt}%
\setlength{\fboxsep}{5pt}%
\ovalbox}%
\MakeFramed{\advance\hsize-\width \FrameRestore}}%
{\endMakeFramed}


\author{Richard L. Sezov, Jr. }
\title{Using Liferay Portal}
\date{12/02/2014}

\begin{document}

\pagestyle{empty}

\titleGM

Using Liferay Portal 6.2

by The Liferay Documentation Team 

Copyright \copyright 2014 by Liferay, Inc.\\[2\baselineskip]
ISBN 978-0-578-15433-6\\[2\baselineskip]
This work is offered under the following license: \\[2\baselineskip]
Creative Commons Attribution-Share Alike Unported

\scalegraphics{../../images/creative-commons-88x31.png}

You are free:

\begin{enumerate}[1.]
\item
  to share---to copy, distribute, and transmit the work
\item
  to remix---to adapt the work
\end{enumerate}

Under the following conditions:

\begin{enumerate}[1.]
\item
  Attribution. You must attribute the work in the manner specified by
  the author or licensor (but not in any way that suggests that they
  endorse you or your use of the work).
\item
  Share Alike. If you alter, transform, or build upon this work, you may
  distribute the resulting work only under the same, similar or a
  compatible license.
\end{enumerate}

The full version of this license is here:

\href{http://creativecommons.org/licenses/by-sa/3.0}{http://creativecommons.org/licenses/by-sa/3.0}

This book was created out of material from the \href{https://dev.liferay.com}{Liferay 
    Developer Network} website. Where the content of this book and the site differ, 
the site is more up to date. 

\clearpage

\frontmatter
\pagestyle{plain}
\pagenumbering{roman}
\chapterstyle{VZ14}

\tableofcontents

\chapter{Preface}

Welcome to the world of Liferay Portal! This book was written for anyone who has
any part in setting up, using, or maintaining a web site built on Liferay
Portal. For the end user, it contains everything you need to know about using
the applications included with Liferay. For the administrator, you'll learn all
you need to know about setting up your site with users, sites, organizations,
and user groups, as well as how to manage your site's security with roles. For
server admins, it guides you step-by-step through the installation,
configuration, and optimization of Liferay Portal, including setting it up in a
clustered, enterprise-ready environment. Use this book as a handbook for
everything you need to do to get your Liferay Portal installation running
smoothly, and then keep it by your side as you configure and maintain your
Liferay-powered web site.

\section{What's New in the Second Edition}

Liferay Portal visually changed a lot in the 6.2 release. This is reflected in 
all the updated screenshots in this book. Of course, everything from the old
book has been updated to reflect the release of Liferay Portal 6.2. We also
have complete coverage of Liferay's new features, such as web content folders, 
the Recycle Bin, Application Display Templates, and more. 

There is, of course, coverage of Liferay Marketplace as well as the Plugin
Security Manager. Mobile devices are first-class citizens in Liferay Portal 6.2, 
and we cover how to serve sites to mobile devices, using responsive layouts 
and mobile device rules. We cover the new, split Control Panel UI, which is 
much more convenient, especially for those who want to grant access to content 
functionality, but not administration of the overall portal. We also cover the 
enhancements to staging and the many new settings for sites, including 
membership management, site hierarchies, and more. 

The new release is feature-packed, and so the documentation must be also: we
have over 700 pages of goodness awaiting you in these pages. 

\section{Conventions}

The information contained herein has been organized in a way that makes it easy
to locate information. The book has two parts. The first part, 
\textit{Using Liferay Portal}, describes how to configure and use a freshly 
installed Liferay Portal. The second part, \textit{Deploying Liferay Portal}, 
is for administrators who want to install Liferay Portal and optimize its 
performance. 

Sections are broken up into multiple levels of headings, and these are
designed to make it easy to find information.

Source code and configuration file directives are presented monospaced, as
below.

\begin{verbatim}
If source code goes multi-line, the lines will be \

separated by a backslash character like this.

\end{verbatim}

\textit{Italics} are used to represent links or buttons to be clicked on in a
user interface.

\texttt{Monospaced type} is used to denote Java classes, code, or properties
within the text.

\textbf{Bold} is used to describe field labels and portlets.

Page headers denote the chapters and the section within the chapter.

\section{Publisher Notes}

It is our hope that this book is valuable to you, and that it becomes an
indispensable resource as you work with Liferay Portal. If you need any
assistance beyond what is covered in this book, Liferay offers training,
consulting, and support services to fill any need that you might have. Please
see \href{http://www.liferay.com/services}{http://www.liferay.com/services} for
further information about the services we can provide.

It is entirely possible that some errors or mistakes made it into the final
version of this book. Any issues that we find or that are reported to us by the
community are documented on the Liferay Developer Network website.\footnote{\href{https://dev.liferay.com}{https://dev.liferay.com}} 

As always, we welcome any feedback. If there is any way you think we
could make this book better, please feel free to mention it on our
forums or in the feedback on the Liferay Developer Network site. You can also
use any of the email addresses on our Contact Us
page.\footnote{\href{http://www.liferay.com/contact-us}{http://www.liferay.com/contact-us}}
We are here to serve you, our users and customers, and to help make your
experience using Liferay Portal the best it can be.

\mainmatter

\pagestyle{headings}


\end{document}
